% -*- coding: utf-8; mode: latex; -*-
%
% FIT2019 向け LaTeX クラスファイル
% https://github.com/trueroad/FITpaper-class
%
% デバッグ用ファイル
%
% Copyright (C) 2018, 2019 Masamichi Hosoda.
% All rights reserved.
% 
% Redistribution and use in source and binary forms, with or without
% modification, are permitted provided that the following conditions
% are met:
%
% * Redistributions of source code must retain the above copyright notice,
%   this list of conditions and the following disclaimer.
%
% * Redistributions in binary form must reproduce the above copyright notice,
%   this list of conditions and the following disclaimer in the documentation
%   and/or other materials provided with the distribution.
%
% THIS SOFTWARE IS PROVIDED BY THE COPYRIGHT HOLDERS AND CONTRIBUTORS "AS IS"
% AND ANY EXPRESS OR IMPLIED WARRANTIES, INCLUDING, BUT NOT LIMITED TO, THE
% IMPLIED WARRANTIES OF MERCHANTABILITY AND FITNESS FOR A PARTICULAR PURPOSE
% ARE DISCLAIMED.
% IN NO EVENT SHALL THE COPYRIGHT HOLDER OR CONTRIBUTORS BE LIABLE
% FOR ANY DIRECT, INDIRECT, INCIDENTAL, SPECIAL, EXEMPLARY, OR CONSEQUENTIAL
% DAMAGES (INCLUDING, BUT NOT LIMITED TO, PROCUREMENT OF SUBSTITUTE GOODS
% OR SERVICES; LOSS OF USE, DATA, OR PROFITS; OR BUSINESS INTERRUPTION)
% HOWEVER CAUSED AND ON ANY THEORY OF LIABILITY, WHETHER IN CONTRACT, STRICT
% LIABILITY, OR TORT (INCLUDING NEGLIGENCE OR OTHERWISE) ARISING IN ANY WAY
% OUT OF THE USE OF THIS SOFTWARE, EVEN IF ADVISED OF THE POSSIBILITY OF
% SUCH DAMAGE.
%

\documentclass{FITpaper}

\usepackage{layouts}

% 和文タイトル
\jtitle{FIT2019向け\LaTeX クラスファイルのデバッグ}

% 欧文タイトル
\etitle{Debugging \LaTeX~Class File for FIT2019}

% 著者数:著者の数だけ c を書く
\authors{ccccc}

% 和文著者名:著者名間に & を書く
% 所属番号を \affmark でつける
\jauthors{%
  〇〇 〇〇\affmark{1} &
  〇〇 〇〇\affmark{1} &
  〇〇 〇\affmark{2} &
  〇〇 〇\affmark{2} &
  〇〇 〇\affmark{2}}

% 欧文著者名:著者名間に & を書く
\eauthors{%
  Foo Bar Baz &
  Foo Bar Baz &
  Foo Bar Baz &
  Foo Bar Baz &
  Foo Bar Baz}

% 所属名
% \affmark でつけた所属番号毎に指定
\afftext{1}{〇〇〇〇〇〇〇〇〇〇〇〇〇〇〇〇〇〇〇〇〇〇〇〇〇〇.
Foo bar baz boo bar baz Laboratories, Foo Corporation.}
\afftext{2}{〇〇〇〇〇〇〇〇〇〇〇〇〇〇〇〇〇〇〇〇〇〇〇〇〇〇.
Foo bar baz boo bar baz Department, Foo University.}

\begin{document}

\maketitle

\section{文字ズレ確認}

□□□□□□□□□■□□□□□□□□□■□□□□□□□□□■
□□□□□□□□□■□□□□□□□□□■□□□□□□□□□■
□□□□□□□□□■□□□□□□□□□■□□□□□□□□□■
□□□□□□□□

本文の文字ズレ確認用。
行間で左右の位置が一致していることを確認。
特に最終行と他の行の間。

\section{バージョン}

\makeatletter
FITpaper.cls: \csname ver@FITpaper.cls\endcsname\par
jlreq.cls: \csname ver@jlreq.cls\endcsname\par
amsmath.sty: \csname ver@amsmath.sty\endcsname\par
\ifx\luatexversion\@undefined
newtxmath.sty: \csname ver@newtxmath.sty\endcsname\par
\else
luatexja.sty: \LuaTeXjaversion ~/ \csname ver@luatexja.sty\endcsname\par
luatexja-preset.sty: \csname ver@luatexja-preset.sty\endcsname\par
fontspec.sty: \csname ver@fontspec.sty\endcsname\par
unicode-math.sty: \csname ver@unicode-math.sty\endcsname\par
\fi
\makeatother

\section{長さ}

1 inch = 72.27 pt = 25.4 mm = 72 bp

\subsection{フォント}

\subsubsection{設定値}

\newlength{\myzh}
\setlength{\myzh}{1\zh}
\textbackslash zh:
\printinunitsof{pt}\prntlen{\myzh},
\printinunitsof{mm}\prntlen{\myzh},
\printinunitsof{bp}\prntlen{\myzh}.

\newlength{\myzw}
\setlength{\myzw}{1\zw}
\textbackslash zw:
\printinunitsof{pt}\prntlen{\myzw},
\printinunitsof{mm}\prntlen{\myzw},
\printinunitsof{bp}\prntlen{\myzw}.

\textbackslash baselineskip:
\printinunitsof{pt}\prntlen{\baselineskip},
\printinunitsof{mm}\prntlen{\baselineskip},
\printinunitsof{bp}\prntlen{\baselineskip}.

\subsection{縦}

\subsubsection{設定値}

\textbackslash topmargin:
\printinunitsof{pt}\prntlen{\topmargin},
\printinunitsof{mm}\prntlen{\topmargin}.

\textbackslash headheight:
\printinunitsof{pt}\prntlen{\headheight},
\printinunitsof{mm}\prntlen{\headheight},
\printinunitsof{bp}\prntlen{\headheight}.

\textbackslash headsep:
\printinunitsof{pt}\prntlen{\headsep},
\printinunitsof{mm}\prntlen{\headsep}.

\textbackslash topskip:
\printinunitsof{pt}\prntlen{\topskip},
\printinunitsof{mm}\prntlen{\topskip},
\printinunitsof{bp}\prntlen{\topskip}.

\textbackslash textheight:
\printinunitsof{pt}\prntlen{\textheight},
\printinunitsof{mm}\prntlen{\textheight},
\printinunitsof{bp}\prntlen{\textheight}.

\textbackslash footskip:
\printinunitsof{pt}\prntlen{\footskip},
\printinunitsof{mm}\prntlen{\footskip}.

\subsubsection{上}

\newlength{\mytop}
\setlength{\mytop}{\topmargin}

\addtolength{\mytop}{1in}

紙上端⇔ヘッダ上端:
\printinunitsof{mm}\prntlen{\mytop}.

\addtolength{\mytop}{\headheight}

紙上端⇔ヘッダ下端:
\printinunitsof{mm}\prntlen{\mytop}.

\addtolength{\mytop}{\headsep}

紙上端⇔ボディ上端:
\printinunitsof{mm}\prntlen{\mytop}.
←上マージン30 mm(目安)

\subsubsection{上下間}

ボディ高さ:
\printinunitsof{mm}\prntlen{\textheight}.

\subsubsection{下}

\newlength{\mybottom}
\setlength\mybottom{\paperheight}

\addtolength{\mybottom}{-\mytop}
\addtolength{\mybottom}{-\textheight}

紙下端⇔ボディ下端:
\printinunitsof{mm}\prntlen{\mybottom}.
←下マージン25 mm(目安)

\addtolength\mybottom{-\footskip}

紙下端⇔フッタ下端:
\printinunitsof{mm}\prntlen{\mybottom}.

\subsection{横}

\subsubsection{設定値}

\textbackslash oddsidemargin:
\printinunitsof{pt}\prntlen{\oddsidemargin},
\printinunitsof{mm}\prntlen{\oddsidemargin}.

\textbackslash textwidth:
\printinunitsof{pt}\prntlen{\textwidth},
\printinunitsof{mm}\prntlen{\textwidth},
\printinunitsof{bp}\prntlen{\textwidth}.

\textbackslash columnsep:
\printinunitsof{pt}\prntlen{\columnsep},
\printinunitsof{mm}\prntlen{\columnsep}.

\textbackslash linewidth:
\printinunitsof{pt}\prntlen{\linewidth},
\printinunitsof{mm}\prntlen{\linewidth},
\printinunitsof{bp}\prntlen{\linewidth}.

\subsubsection{左}

\newlength{\myleft}
\setlength{\myleft}{\oddsidemargin}

\addtolength\myleft{1in}

紙左端⇔左カラム左端:
\printinunitsof{mm}\prntlen{\myleft}.
←左マージン20 mm(目安)

\subsubsection{左右間}

左カラム左端⇔右カラム右端:
\printinunitsof{mm}\prntlen{\textwidth}.

左右カラム間:
\printinunitsof{mm}\prntlen{\columnsep}.
←カラム間7 mm(目安)

\subsubsection{右}

\newlength{\myright}
\setlength{\myright}{\paperwidth}

\addtolength\myright{-\myleft}
\addtolength\myright{-\textwidth}

紙右端⇔右カラム右端:
\printinunitsof{mm}\prntlen{\myright}.
←右マージン20 mm(目安)

\section{フォント}

\subsection{本文}

明朝体Serif
\textsf{ゴシック体San Serif}
\texttt{等幅Type Writer}

\subsection{数式}

\begin{align}
  \frac{\pi}{2} &=
  \left( \int_{0}^{\infty} \frac{\sin x}{\sqrt{x}} dx \right)^2 \nonumber \\
  &= \sum_{k=0}^{\infty} \frac{(2k)!}{2^{2k}(k!)^2} \frac{1}{2k+1} \nonumber \\
  &= \prod_{k=1}^{\infty} \frac{4k^2}{4k^2 - 1} \nonumber
\end{align}

\acknowledgment{%
  □□□□□□□□□■□□□□□□□□□■□□□□□□□□□■
  □□□□□□□□□■□□□□□□□□□■□□□□□□□□□■
  □□□□□□□□□■□□□□□□□□□■□□□□□□□□□■
  □□□□□□□□

  謝辞の文字ズレ確認用。
  行間で左右の位置が一致していることを確認。
  特に最終行と他の行の間。
}

\begin{thebibliography}{9}

\bibitem{one}
  □□□□□□□□□■□□□□□□□□□■□□□□□□□□□■
  □□□□□□□□□■□□□□□□□□□■□□□□□□□□□■
  □□□□□□□□□■□□□□□□□□□■□□□□□□□□□■
  □□□□□□□□□

\bibitem{two}
  参考文献の文字ズレ確認用。
  行間で左右の位置が一致していることを確認。
  特に最終行と他の行の間。

\bibitem{three}
  \textbackslash labelwidth: \the\labelwidth,
  \textbackslash labelsep: \the\labelsep,
  \textbackslash labelwidth + \textbackslash labelsep:
  \newlength{\temptotal}\setlength{\temptotal}{\labelwidth}
  \addtolength{\temptotal}{\labelsep}\the\temptotal,
  \textbackslash leftmargin: \the\leftmargin,
  \textbackslash zw: \the\myzw,
  \textbackslash linewidth: \the\linewidth,
  \textbackslash textwidth: \the\textwidth.

\end{thebibliography}

\end{document}
